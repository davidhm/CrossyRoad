\documentclass{article}
\usepackage[utf8]{inputenc}
\usepackage{hyperref}
\usepackage{textcomp}
\usepackage{graphicx}
\title{Crossy Road \\
		\large Videojocs 2017/18 Q2}
\author{David Hernàndez Morales}

\begin{document}
\pagenumbering{gobble}
\maketitle
\newpage
\pagenumbering{arabic}
\section{Crossy Road}
El joc que s'intenta emular és Crossy Road \textsuperscript{\texttrademark}
publicat (i desenvolupat) el 20/11/2014 per \textit{Hipster Whale} \cite{webCrossy}
\cite{wikipediaCrossy}. A la figura \ref{logo} es pot veure el logo de Crossy Road 
amb l'emblemàtic pollastre que pretèn donar resposta a la pregunta 
\textit{Why did the chicken cross the road?} \cite{preguntaChicken} 

\begin{figure}[h!]
	\includegraphics[width=\linewidth]{Logo.png}
	\caption{Logo del crossy road}
	\label{logo}
\end{figure}

Com es pot veure a la taula \ref{dadesSortida}, el primer llençament va ser en
\textit{iOS}. Amb \textit{Android} un mes després i \textit{Windows Phone}
quasi un any més tard. També hi ha una versió per \textit{tvOS}.

\begin{table}[h!]
	\begin{center}		
		\label{dadesSortida}
		\begin{tabular}{l|l}
		\textbf{Plataforma} & \textbf{Data de sortida} \\
		\hline
		iOS & 20 Novembre, 2014 \\
		Android & 23 Decembre, 2014 \\
		Windows Phone & 1 Maig, 2015 \\
		tvOS & 30 Octubre, 2015		
		\end{tabular}
		\caption{Dates de sortida de Crossy Road.}
	\end{center}
\end{table}

\bibliography{Fonts}
\bibliographystyle{abbrv}
\end{document}